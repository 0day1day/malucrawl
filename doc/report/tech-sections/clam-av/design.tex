
\section{ClamAV Malware Scan}
In order to fully identify the existence of malware inside a webpage, a 
signature based malware detection method must be implemented in the system. 
The most obvious approach was to add anti-virus software to the scanning system, 
in order to enable the classification of URLs and files with non-HTML contents, such 
as binary files and scripts. Therefore ClamAV was used as a step in scanning for this purpose, which is also a low interaction malware detection 
method.

\subsection{Design}
\subsubsection{Introduction to ClamAV}
ClamAV is an open source anti-virus software developed for UNIX, and is 
available to Linux systems. It has a command-line scanner with an open access 
signature database, where the anti-virus engine is in a shared library. 
The software provides both on-demand and on-access scanning on Linux systems, 
where on-access scanning is achieved with a scalable multi-threaded daemon. 
It also has wide support for various file formats, including almost all 
popular compressed files, emails and documents. \\
A report in 2011 rated ClamAV 12th among all publicly released anti-virus 
applications,\cite{shandowserver} where the most impressive part is its fast 
response time and scanning speed. This is a key feature that is particularly useful for the purpose of the system. The processing of large URL
datasets requires considerable computing power, where the scanning process 
occupies most of it. \\
ClamAV also has a third party Python API known as pyClamAV. It is a free 
library implemented in C and binds to the ClamAV's own API, libClamAV. However 
the newest version has moved its focus to pyClamd, which is an interface to Clamav 
daemon instead of libClamAV. Due to the latest updates to ClamAV, some 
key APIs were removed from libClamAV. \\
Overall, the conclusion made was to use ClamAV over other 
anti-virus software for the following reasons: 
\begin{itemize}
\item Open source
\item Large signiture database and high detection rate
\item Linux compatible
\item Python compatible
\item Wide support for various file types
\item Fast response time and scanning speed
\end{itemize}

\subsubsection{ClamAV with HTML crawler}
A program was built to download URLs and perform static analysis conforming to the term --- low interaction. The given URL should always contain 
an HTML file, and after the virus scan the program crawls it in order to extract links to pages 
under the website. After that the extracted links are again downloaded and 
scanned with ClamAV. This time the file types are uncertain, therefore the program was given a max file size to download as well as a time limit in case of a 
slow or broken connection. This is a vital step to avoid downloading huge files such as 
high definition videos as well as without a file length, for example, a live radio stream.

Links to other domains are ignored, because only the investigation 
of the given URL's domain is concerned. Duplicate links are removed, and the 
program avoids downloading the same page twice or more, as the root page needs 
to be processed twice: once for crawling and once for virus scan.

Every ClamAV scan is performed by a separate Celery task, which means the 
number of tasks should be $n+1$ where n is the number of links interested and 
the other time is for the root page.  

\subsection{Implementation}
\subsubsection{Libraries}
The following libraries are used in this Python application:
\begin{itemize}
\item {\bf pyClamd} This is the ClamAV daemon library used for URL content 
scanning. Compared to ordinary ClamAV, a daemon is faster and 
multi-threaded. It listens on Unix local socket or TCP socket to work, which 
means it supports remote scanning. However in the program only local socket is 
used as remote scanning is not necessary. 

\item {\bf requests} The network library used for HTTP inquiries. In the 
program it is used for downloading the root HTML page as well as links inside 
of it. 

\item {\bf lxml.html} lxml is a Python XML library, it creates a 
binding to the C libraries libxml2 and libxslt.\cite{lxml} lxml.html is a 
package specifically designed for HTML parsing, and acts as an HTML crawler 
which extracts links. 

\item {\bf eventlet.timeout} Eventlet is a Python library that enables 
concurrent networking. The $timeout$ package provides a universal timeout 
solution achieved with green threads, and it is this package that makes the time limit 
for downloads realizable. 
\end{itemize}

\subsubsection{Program implementation}
The program flow is explained as following:
\begin{enumerate}
\item A Celery task is created with the crawler function given a URL as a 
parameter. The URL is then downloaded via requests. 

\item The HTML parser parses the downloaded content as an HTML file, then makes 
all links in it absolute, which is achieved with the base URL provided. 

\item All links are extracted with the library function {\em html.iterlinks()}, 
results are returned as a list for easy manipulation. Thus duplicate links and 
those with different domain names are removed.

\item For each link the download function is called with a separate Celery 
task. A get request which only acquires HTTP headers is sent, such that the file size can be determined by the 
{\em content-length} field. If the file length is within a limit, just before 
the download begins a green thread with a {\em Timeout} is created which throws 
an exception if the download time is too great. Zero is returned if any exception 
occurs before or during downloading, which includes but not limited to request 
exceptions (such as error response code) and time outs. After successful 
downloading a clamd scan to the content is called, the return value should be 
either {\em None} or the virus description message. 

\item The final list of malware detected is merged by the results 
returned from the scan results of all downloaded links as well as the root 
page, which returns to the main system.  
\end{enumerate}
