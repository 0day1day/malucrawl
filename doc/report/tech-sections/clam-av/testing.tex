\subsection{Testing}
We test the program's individual functions separately. The sequence of 
testing is as follwing:
\begin{enumerate}
\item Downloading and parse of HTML. 
\item Extraction of links and make them absolute.\\
This is tested by printing all links as absolute from a given HTML and check 
manually. 
\item Removal of duplicate link and those links to other domains. 
\item Downloading of an arbitrary URL. \\
We test this function with download testing websites such as \\
\verb`http://www.thinkbroadband.com/download/`\\
The tests with various file 
sizes passed without errors and those links exceed size or time limits are 
ignored. 
\item Clamd scan of the downloaded contents. 
\end{enumerate}
After that we test the whole program with benign websites which provide 
testing virus files. 
These files used for testing are not real virus however follw the patterns 
that malwares would behave. \\
One of the websites we used is Eicar (
\verb`http://www.rexswain.com/eicar.html`). This website provides a small test 
virus in three compressed level: plain text, zipped and double-zipped, who's 
download links are in the same page. Here is the result we obtained after 
crawling the webpage:
\begin{verbatim}
[('http://www.rexswain.com/eicar.com', 
{'stream': ('FOUND', 'Eicar-Test-Signature')}), 
('http://www.rexswain.com/eicar.zip', 
{'stream': ('FOUND', 'Eicar-Test-Signature')}), 
('http://www.rexswain.com/eicar2.zip', 
{'stream': ('FOUND', 'Eicar-Test-Signature')})]
\end{verbatim}
Consequently we successfully finished the unit testing of ClamAV scanner. 
