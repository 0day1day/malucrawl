\subsection{Testing}
The program's individual functions were tested separately. The sequence of 
testing was as following:
\begin{enumerate}
\item Downloading and parse of HTML.
\item Extraction of links and make them absolute.
This is tested by printing all links as absolute from a given HTML and check 
manually.
\item Removal of duplicate link and those links to other domains.
\item Downloading of an arbitrary URL.
This function was tested by downloading testing websites such as
\verb`http://www.thinkbroadband.com/download/`
The tests with various file 
sizes passed without errors and those links that exceeded the size or time limits are 
ignored. 
\item Clamd scan of the downloaded contents. 
\end{enumerate}
\paragraph{}
After that the whole program was tested with benign websites which provide 
testing for various viruses. These files used for testing were not real viruses however anti-virus packages recognize these signatures 
as part of their test suite. 
One of the websites used was Eicar\cite{eicar}. This website provides a small test 
virus in three compressed levels: plain text, zipped and double-zipped, with 
download links in the same page. The results obtained after 
crawling the web page are presented below:
\begin{verbatim}
[('http://www.rexswain.com/eicar.com', 
{'stream': ('FOUND', 'Eicar-Test-Signature')}), 
('http://www.rexswain.com/eicar.zip', 
{'stream': ('FOUND', 'Eicar-Test-Signature')}), 
('http://www.rexswain.com/eicar2.zip', 
{'stream': ('FOUND', 'Eicar-Test-Signature')})]
\end{verbatim}
Consequently the unit testing of the ClamAV scanner was successful. 
