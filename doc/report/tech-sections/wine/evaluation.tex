\subsection{Evaluation of Wine Explorer}
Wine Explorer is indeed a feasible approach for high interaction malware 
detection. As Wine is a compatibility layer instead of virtual machine or 
emulator, it enables IE to execute with minimal resource consumption. 
An application run by Wine is also 
treated as ``first class citizens''\cite{wineperformance} and 
permissions for it are not 
restricted. The creation of a Wine prefix is also impressively fast, which 
means low overhead for honeypot resets. 
\paragraph{}
However disadvantages still exists. We explained that the Wine prefix is treated
as a sandbox, and the reason is almost all Windows applications do not operate 
with Linux environment. Imagine if attackers predicted the Wine Explorer 
environment, then they could easily develop Windows applications which inject 
Linux shell commands, which means, they will have access to the file system 
outside of Wine prefix. Also, the suspicious behaviour detection functions of 
this program is really naive which cannot cover all situations. An example 
could 
be a downloaded virus that is able to delete itself before the scan begins. 
During that time it is sufficient to perform illegal behaviours such as 
sensitive information collection. The modification of system 
registry is also not considered in this application. According to the project 
schedule we are unable to achieve everything perfectly due to the approaching 
deadline. 
