\subsection{Evaluation of Wine Explorer}
Wine Explorer is indeed a feasible approach for high interaction malware 
detection. As Wine is a compatibility layer instead of virtual machine or 
emulator, the Internet Explorers running under it can execute in full speed, 
where the computation resource consumption is minimal compared to what 
a virtual machine or emulator require. The applications run by Wine are also 
treated as first class citizens such that permissions for them are not 
restricted. The creation of a Wine prefix is also impressively fast, which 
means the overhead for honeypot reset is also minimal. \\
However disadvantages still exists. We explained that we treat a Wine prefix 
as a sandbox, and the reason is almost all Windows applications do not operate 
with Linux environment. Imaging if attackers predicted our Wine Explorer 
environment, then they could easily develop Windows applications which injects 
Linux shell commands, which means, they will have access to the file system 
outside of Wine prefix. Also, the suspicious behaviour detection functions of 
this program is really naive which cannot cover all situations we can think 
of. An example could
be a downloaded virus which is able to delete itself before the scan begins. 
In that time it is sufficient to perform illegal behaviours such as sensitive 
information collection. The modification of system 
registry is also not considered in this application. According to the project 
schedule we are unable to achieve everything perfectly due to approaching 
deadline. 
