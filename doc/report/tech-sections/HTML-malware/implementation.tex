
\subsection{Implementation}
The fact that there is an unclear definition of a malicious webpage makes the implementation of a malware detection system quite challenging. 	  
	

\subsubsection{Frequency of the keyword}
 
In this approach HTML malware detection takes a URL and a keyword as an input. The HTML crawler scans the main webpage and counts the frequency of the given keyword in that webpage. Also, the crawler scans the contents of the webpage for all the hyperlinks in the main webpage. The scanner then computes the ratio as showed below;

Ratio = Total number of keywords within the main web page / Total number of keywords in the inner hyperlinked web page 

There are three possible ratio result as shown below. 

Case (1): Ratio = 0

The zero ratio indicates the keyword does not exist in the main web page. Therefore the attackers used, a trendy keyword in the components of the given URL to misdirect the user to the malicious webpage

Case (2): Ratio = infinity

According to the ratio equation, there is no instance of a given keyword within the inner hyperlinked web page. Therefore, the contents of the inner hyperlink web page are not relevant to the content of the main web page. Hence, the given URL is deemed malicious.

Case (3): Ratio >= 1

That means the number of trendy keyword appears in the main page is more than or equal to the total number of trendy keyword in the web page inner links. In the other words, the contents of the inner links web page is less relevant to the contents of the main web page. In this case the given URL has identified as a malicious URL. 

Case (4): 0 < Ratio < 1

If the ratio is closer to zero that means the number of keywords appearing in the inner hyperlinked web pages are greater than the number of the keywords on the main web page. Hence, a smaller ratio indicates a higher probability that a given URL is benign. Accordingly, a large ratio (ratios close to one) has a greater probability that the given URL is malicious. 

\paragraph{} 
The prima facie weakness of this approach is the risk that a malware writer will insert many instances of trendy keyword terms within the inner hyperlinked web page and instead add malicous hyperlinks from within the inner hyperlinked webpage to another malicious websites or perhaps even from a further removed linked website. Thus, a malware writer could use an iterative process to manipulate the ratio result.To avoid this problem, there is similarly a iterrative process for checking and calculating a ratio at each instance when there are links to another webpage. Hence, the ratio may not inidicate a malicious web page at the first few iterations but then identify a malicious website after checking the last inner hyperlinks. Adapting this method has helped to overcome the major problem for designing HTML malware scanner, because it is independent of the malware definition. In additions, the method is fairly fast.  
