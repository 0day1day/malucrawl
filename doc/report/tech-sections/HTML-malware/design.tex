\section{HTML Analysis Malware Scanner}

HTML Malware Scanner aims to identify the infected HTML webpage. A major problem in designing this scanner is the lack of a clear definition of a malware webpage. Web pages containing illegal content are considered as a malware webpage. Also, the contents of some webpages might appear innocent but some links in those webpages may direct the user to an unauthorised website. The latter is the most popular method attackers use to target internet users. Therefore, finding a solution for this scanner is challenging. Attackers use trendy keywords to increase the probability of a successful attack. In this project, we seek to observe the relationship between the trendy keywords and infected webpage.

\subsection{Design}

The design of the HTML scanner is based upon the frequency of a repeated trendy keyword on a webpage. The HTML scanner accepts two inputs, a URL and a keyword. The scanner searches the webpage to find the quantity of the given keywords. In addition, the scanner identifies all hyperlinks within the given webpage and in turn searches the content of each linked URL to establish the quantity of the given keywords contained within the linked webpages. The frequency of the keywords within the main URL webpage is divided by the sum of frequency of the keywords within the linked webpages in order to calculate a ratio. The HTML scanner used the ratio to decide whether a given URL is malicious or not. Sections 2 illustrates the implementation of the HTML malware scanner and provides details of this ratio calculation.

There is iteration on finding the inner URLs in each web page. This iteration carried out until there is no hyperlinks in the web page of inner URLs. Hence, the malware writers cannot bypass the scanner by inserting more keywords in the inner hyperlinks. 
