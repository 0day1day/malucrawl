\subsection{Design}
Intro
Overall Design
Phases of the system:
Trend Analysis, Search Engine, Multiple Malware Scanning Modules, Storage, Display/Reporting.
Multiple Machine System
Task dispatching

determine that a URL has at one point returned a malicious page

\subsection{Implementation}
This section describes how to use a set of workers to determine trending topics and sites relating to those topics

% python, pip and PyPi
% HTML/XML, RSS and JSON parsing
% OAuth2
% Celery basics

\subsubsection{Celery Task Queues}
Celery provides a basic interface for running tasks on a cluster of workers. This could be one worker per core of a machine, or of many machines on a network.

\subsubsection{Overall Design}
Each result from the ``keyword finding'' task should result in a new ``search engine'' task.
Each result from the ``search engine'' task should result in a new ``malware finding placeholder'' task
Each result from the ``malware finding placeholder'' should be printed to STDOUT by the master node.

It is expected that the ``search engine'' will return hundreds of results, as such the Celery chunking api should be investigated to optimise resource usage. \verb`http://docs.celeryproject.org/en/latest/userguide/canvas.html#chunks`
Future Extension
The system should be designed so that it is possible to avoid repeating the ``same'' task too often. Eg a ``search engine'' task for the same keyword, or ``malware finding'' task for the same URL should only be repeated once in any given time period.

To implement this a set of keywords and URLs should be maintained. Each item should be dropped from the set after a certain time. This could be in a database eg. the existing Redis instance used for the result backend:

\verb/{"task:(keyword|url):#{data}":{"result":taskid, "requested":time_stamp}}/

The simplest solution is that: each time a task is called it checks in Redis for its task key, eg ``task:keyword:foo'' and returns the result at ``taskid'' if the time is recent enough. Otherwise it adds itself to Redis and returns normally.

Despite being conceptually to check the sets each time, this is a waste of bandwidth. As such the controlling node should be able to check in most cases.

Note: In the end it might not be worth it, as we will only be checking the keyword sources as often as they claim to update and checking for duplicate scanned URLs could be done later in the stack eg. before they are dispatched to the Windows vms.
Keyword Finding
Each ``keyword finding'' task should be able to query some external service and determine a set of keywords. These keywords can then be dispatched to other tasks.

Libraries available from PyPi likely to help with this include:

\begin{itemize}
    \item requests A simple API for making HTTP requests
    \item feedparser For parsing RSS feeds
    \item lxml.html and BeatuifulSoup. For parsing and interacting with ``real world'' HTML
\end{itemize}

It is important to note that further possibilities for keyword determination should be researched. The sources suggested here are the ones I think will be the easiest to interface to, but might not be the sources malware distribution networks actually use!

\subsubsection{Twitter}
The twitter api is available in multiple versions, 1 and 1.1. Currently 1 is deprecated and it is recommended that new applications build against 1.1.  The disadvantage of the 1.1 API is that it requires a ``User Context'', ie all requests must be authenticated based on Application-User key.

The Twitter API provides access to global trending topics at

\verb`https://api.twitter.com/1.1/trends/place.json`

Documented at \verb`https://dev.twitter.com/docs/api/1.1/get/trends/place`

The keyword information is available at [0].trends[].name.

Topics with a \# infront are Twitter ``hashtags'', and are usually camel cased. The system should be able to convert from camel cased to space delimited. As in \verb`http://commons.apache.org/lang/api-3.0.1/org/apache/commons/lang3/StringUtils.html#splitByCharacterTypeCamelCase%28java.lang.String%29`

Eg. \#BrazilWantsOneDirection -> Brazil Wants One Direction
\#XFactorPHFinale -> X Factor PH Final

In the case of all capi
    \item Use simpletals eg NIALLSAYHITOTURKEY, it's probably best to drop it.

\subsubsection{RSS}
A set of RSS feeds should be collected and parsed, ``feedparser'' is likely to be useful for this.

Each item from a feed should be entered into a corpus, to calculate keywords from items in the corpus algorithms such as TFIDF should be used. The natural language library ``metanl'' is likely to be helpful.

Note that each feed can be processed separately. Ie one corpus per feed.

\subsubsection{A note on the Sun}
The Sun provides a JSON API for the top 200 most commented on articles at \verb`http://bootstrap.thesun.fyre.co/api/cust/ni/todays_hottest/200.json`. In most cases the content of this JSON file can be treated exactly the same as an RSS feed however care should be taken to deal with the dodgy encoding of some of the returned headlines: ``Terry a â\\u0080\\u0098liarâu\\0080\\u0099'' should be ``Terry a \"liar\"''.

\subsubsection{Even More Sources}
Examples include:
\begin{itemize}
    \item Link aggregation sites such as Reddit and Digg.
    \item Search engine common searches and trends such as Google, Bing and Yahoo trends services.
    \item Network logs such as the campus search trends. It should be noted that this data is very unlikely to be accessible by malware distribution networks and might allow us to see trends before they do.
\end{itemize}

\subsubsection{Search Engine}
To determine URLs to investigate a search engine of some sort is required, the best I have found that can be used by automata is http://dogpile.co.uk.

A handy code snippet for retrieving URLs from dogpile:

map(lambda link: link.get('href'),lxml.html.fromstring(requests.get("http://www.dogpile.co.uk/search/web?q=foo").text).cssselect(".webResult .resultDisplayUrl"))

Because each of the links we want is wrapped in a ``ClickHandler'' \verb`http://cs.infospace.com/ClickHandler.ashx?du=http%3a%2f%2fwww.ietf.org%2frfc%2frfc3092.txt&ru=http%3a%2f%2fwww.ietf.org%2frfc%2frfc3092.txt&ld=20121006&ap=10&app=1&c=uk.dogpl&s=dogpileuk&coi=239138&cop=main-title&euip=94.193.128.47&npp=10&p=0&pp=0&pvaid=1351848139fc4226b6b6a0f6e0eed58b&ep=8&mid=9&hash=22AE450C7F5B9FA7C51FE2CEA7841093`

The URLs returned must be parsed, eg using ``urlparse'' to retrieve the unwrapped URL available in the ``du'' or ``ru'' GET parameter, in this case\cite{rfc3092}