\subsection{Design}

As one of the selection of high-interaction malware scanning
methods available for use in the malucrawl framework,
Capture-HPC\cite{capture-hpc} is a way of
realistically emulating a user browsing to a given URL in a real web browser.
Capture-HPC is designed as a client honeypot, where in contrast to the usual and
more common form of passive honeypots that simply wait for incoming attacks,
goes out and requests content from the Internet and attempts to detect any
malicious activity caused by the content. Capture-HPC was developed at Victoria
University in Wellington NZ and is maintained be the HoneyNet project(CITE).

\subsubsection{Capture-HPC in Detail}

%TODO: Detailed description of what capture-hpc does and how it works.
Capture-HPC uses virtualisation to simulate user browsing, using a web browser
running in a monitored and externally controlled Virtual Machine. There are two
distinct pieces of software used to do this, a client to run the web browser on
the virtual machine and monitor for malicious activity, and a server to
orchestrate URL scanning between however many clients are being used and to
reset client VMs when malicious activity is detected. The client and server
components communicate over a network connection.

The Capture-HPC client is a C++ program that runs on Windows XP, Vista, or 7.
It's role is to automatically browse the web and to monitor the VM for any
suspicious activity, reporting the outcome rendering URLs to the server. The
client is capable of driving a number of applications using URLS, from web
browsers such as Microsoft Internet Explorer and Mozilla Firefox, to downloading and opening
Microsoft Word documents and Adobe PDFs. For the purposes of the project, the
use of Internet Explorer is prioritised as a starting point for malware scanning
that can be easily extended to cover the other applications. To detect any
malicious activity that may be caused by visiting a website, or opening a
downloaded file, a collection of kernel drivers are used. The kernel drivers
watch filesystem, network and process activity at a low level, ensuring that
unusual activity can be detected despite attempts from any malware to conceal
its actions. This isn't a technique that works in general on a PC where the user
may be performing any number of other actions whilst using the machine, but
works well in the controlled environment provided by the virtual machine. To
stop the small number of standard processes and filesystem changes made during
normal operation of the operating system and client applications triggering the
malware detection, Capture-HPC uses a set of "exclusion lists" that specify
actions for each monitoring tool that should be considered as safe. It is also
possible to specify precisely which actions should cause the monitoring tool to
report suspected malicious activity.

The Capture-HPC server program that controls the clients is a Java program that
will run on most platforms (platform compatibility is limited by the VM revert
script detailed below). The role of the server is to communicate the exclusion
lists and URLs to be tested to the clients, and then collect the results of the
analysis. The server also uses and external script to reset the virtual machine
if malicious activity is detected by the client. The server also resets the
client if it does not respond to keepalive packets, so the client does not get
stuck on a URL that errors. There exists two options for supplying URLs to the
Capture-HPC server, using a plaintext input file with one URL on each line (the
line can also contain options controlling which client application is used to
open the URL) or a MySQL database with a schema provided with the application.
The server stores the results in text files labelled for safe, error, and
progress by default, with additional detail for suspicious URLs stored in a file
named with the suspicious URL (URL encoded). This is not ideal for use within
the project, but fortunately Capture-HPC will store all results in the
relational database if one is configured. The use of a MySQL table for a queue
could be considered an anti-pattern as discussed in (CITE), but is the best of
the options available for providing Capture-HPC with data, and retrieving the
results. Unfortunately the Capture-HPC server also has to be restarted to
consume new URLs from the database, with any finished URLs that are left in the
database processed again. The Capture-HPC server is configured using an XML
configuration file that can be found in Appendix TODO.

The version of Capture-HPC to be used in the project supports VMware as its
virtualisation platform, and the ECS VM infrastructure also runs on VMware
products. Most VMWare products offer an API for programmatically interacting
with the Virtual Machines being hosted, known as VIX, and Capture-HPC uses this
API to reset suspicious VMs. The API however only has bindings in perl and C,
so a bridging program written in C is supplied with Capture-HPC and must be
compiled independently of the server. Each virtual machine being used as a
Capture-HPC client needs to have the client software installed, and then a
snapshot taken. A VM Snapshot is a method of backing up the VM in manner that
preserves all information about the state of the VM, including the memory,
meaning that the state of the machine can quickly be restored to its original
after a malware detection. The bridging utility simply reverts the VM to the
first snapshot it can find, and then re-starts the Capture-HPC client.

\subsubsection{How Capture-HPC is used in the project}

Capture-HPC is the most accurate simulation of user browsing used by the project
as a malware scanner, and has a very low URL throughput thanks to each URL
needing to be rendered in a web browser. The throughput is further reduced be
the need to rebuild the VM every time malicious activity is detected. Therefore
it is necessary to limit the number of URLs that Capture-HPC is required to
process. As discussed previously, limiting of the URLs processed by Capture-HPC
is done by a classification system, and only URLS that are strongly suspected to
harbour malware are submitted for scanning. In addition to the confidence factor
that each malware analyser in the framework must return, Capture-HPC is also
capable of returning a description of the actions taken by a malicious page, and
these details are returned to the framework for optional in-depth analysis.

\subsubsection{Security Considerations}

As Capture-HPC renders web pages, there is a significant risk that any malicious
content will take over the VM and try to exploit other machines connected to it
or send spam. To mitigate this threat, a set of preventative measures are taken
as shown in the diagram below.

AWESOME SECURITY ARCH DIAGRAM

To secure the virtual machine(s) running the Capture-HPC client, a Linux
firewall is placed between the VMs and the network, with the client VMs only
able to communicate with the outside network via the firewall. The firewall
blocks all ports by default, only proxying a small selection of ports and
forwarding no ports. To allow the Capture-HPC client to request web pages, DNS
queries are proxied through the firewall, with all requests logged. All HTTP and
HTTPS traffic is also proxied and logged using a rolling log so any suspicious
activity can be investigated. A self-signed SSL certificate will be installed on
the client VM so that HTTPS connections still appear valid despite interception.
A small selection of ports will also be open into and out of the firewall to
allow control of Capture-HPC and reporting of results.

\subsection{Implementation}

\subsubsection{Deployment of Secure Architecture In ECS}

%TODO: compiling client was difficult, but can port image to other platforms,
%re-use capture binary

When deploying Capture-HPC in ECS to form part of our prototype deployment of
the framework, a small group of VMs were used host the security architecture
discussed above. The gateway VM had RHEL 6 installed for stability and security,
with iptables used to provide a firewall. An annotated version of the iptables
rules used for running Capture-HPC is listed in Appendix TODO, explaining the
function of each rule. Squid was used as a HTTP and HTTPS proxy, compiled with
support for dynamically generating certificates for HTTPS websites. To ease 
development on the system, a small tool was written to allow a number of
iptables "profiles" to be created, and then changed using a small shell script,
the source of which can be found in Appendix TODO. The Capture-HPC Server
program was compiled and run from this machine.

The Capture-HPC client was compiled on a Windows XP development VM, and then
deployed to a clean install of Windows XP. It is also possible to run the
Capture-HPC client on a Windows Vista or Window 7 install, but was not tested in
the prototype deployment due to time constraints. The Windows XP install used
for the client had no anti-virus or firewall configured, with Internet Explorer
6 installed to present an easy target for malicious sites.

\subsubsection{ECS Specific Customisations}

The specific network architecture meant that some customisations had to be made
to the standard deployment, and to further reduce risk of malware escaping from
behind the firewall. The ECS VM infrastructure uses VMware vSphere to manage its
ESXi VM servers, and the Capture-HPC server must connect to the vSphere
administration console to revert the VMs when malicious activity is detected.
The VM administration API requires authentication, which uses ECS domain
credentials. The consequence of this is that a domain password must be stored in
a file on the VM running the Capture-HPC server, which has unacceptable
consequences if the Capture-HPC server VM is compromised. Another issue is that
the API is not accessible from the DMZ where the Capture-HPC server is deployed,
again for security reasons.

The solve this issue, a small RPC system was set up using AMQP message queues to
allow a trusted client in a trusted part of the network to connect out to the
server in the DMZ, and the message queue route revert requests out to the
trusted server. The Capture-HPC server only holds fake password details, which
are replaced with the real credentials on the trusted client. A diagram showing
the design of the RPC system is shown below. 

AWESOME REVERT RPC DIAGRAM

\subsubsection{Framework Integration}

Capture-HPC, being written in a combination of Java, C++ and C needs an
interface module before it can be integrated with the python framework.
Capture-HPC can be backed using a database for URLs, which is useful as it
allows us to build Capture-HPC's database format into a Django model which can
be used in the python code to access the database. Each Capture-HPC server has a
celery worker running on the same machine, that communicates with the MySQL
database also running on the same machine, and controls the Capture-HPC
server, which needs to be restarted for each new batch of URLs to be processed.
The database also has to be cleared of URLs, and results reported back to the
framework.

\subsubsection{Exclusion Lists}

The exclusion lists used to determine what activity is judged by the Capture-HPC
clients as malicious also needs to be maintained. The Capture-HPC server is
capable of sending the lists to the clients, allowing easy distribution of new
exclusion lists. The exclusion lists as of time of writing are included in
appendix TODO, and are stored under the project version control, but the lists
created are only tested with Windows XP, so it is quite possible that they will
need to be modified for use under Windows 7 or Windows Vista.

