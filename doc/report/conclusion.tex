\section{Conclusion} 
\subsection{URL Classification}

%need to make sure it is clear that we're talking about the URLs rather than the content?
An unknown malicious URL cannot be identified by systems that use signature-based malware detection methods. As a solution, the designed system includes a classifier in order to detect unknown malicious URLs. The classifier relies on pattern matching algorithm used in the Levenshtien distance and has two classes of data; malicious URLs and clean URLs. An unknown URL enterings to the detection system will be classified based on its similarity to the URLs in the dataset. The result of the classification algorithm provides a confidence rate regarding the URL’s maliciousness. Depending on the confidence rate provided, the classifier will then classify whether the given URL should be subject to a heavyweight or lightweight scanning process. However, where the URL is already held within the database, an immediate result is returned confirming with a one hundred percent confidence rate whether or not the URL is malicious. Because a given URL does not need to be checked by multiple scanners simultaneously and unnecessarily, the classifier creates a more efficient and robust detection system. 
 
\subsection{HTML Scanning}

The HTML malware scanner specifically processes trendy keywords from a search engine and within an infected web page to detect a malicious URL. The HTML scanner has been designed based on the repetitive frequency of the trendy keywords within the main URL web page and within the contents of associated hyperlinked web pages. The scanner is efficient and can easily be extended with sophisticated techniques to produce more accurate results. 
 
\subsection{Literature review}
The research in background knowledges gave us better understanding of current 
technologies about malware distribution over the internet. We realised the 
most used approach is to poison search engines which is called Search Engine 
Optimization (SEO), and in order to achieve this the use of trending terms 
is necessary. The processes of a SEO attack was explained, and a method to 
track down SEO campaigns was also introduced. We looked into details about 
how trending keywords can be applied to generate MFA and malicious websites 
such that they can be found by search engines. The significance of search 
engine intervention was proved by Google during Februrary, 2011. \\
Another report acts as our basics of low and high interaction client 
honeypots. It explained passive and active methods for malware detection 
over the web as well as the differences between low and high interaction 
approaches. The automatic malware collecting system they produced inspired 
us the details about how to implement our system. 

\subsection{Wine Explorer and ClamAV scanner}
Wine Explorer provides a lightweight solution for high interaction malware 
detection. It has advantages over virtual machines and emulators including 
short reset time, fast execution and RAM/disk saving, where the potential 
security threat blocks its way to perfection. The program is also approved to 
execute stably, for instance the correctness of the Wine prefix package is 
guaranteed via hash check which avoids external modifications of it. We 
explained the lacking of complete detection as well, which could be 
implemented if we have more time in this project. 
\paragraph{}
We also provide a signiture-based malware scanning powered by ClamAV. The 
HTML crawler in the same program gives links for ClamAV to scan and an 
overall list of malicious links in a webpage can be produced from the 
application. It has fast scanning and executes concurrently which rises 
the system's throughput to a significant amount where disadvantages exists 
such as unprecise link filtering. An improvement could be a deeper level of 
website crawling, and better link processing which requires further amount 
of time to complete. 


\subsection{Capture-HPC}




Capture-HPC is a high interaction client honeypot, a system designed to actively
search for malware by rendering the content at provided URLs. In addition to
offering a accurate simulation of a user browsing the web Capture-HPC provide
much flexibility, capable of automatically rendering content using a variety of
client programs including web browsers such as Internet Explorer and Firefox,
and the PDF viewer Adobe Reader. It also allows a degree of freedom in choosing
the sensitivity of the malware detection using the exclusion lists, a feature
that also makes the process of porting Capture-HPC to other Windows OSs
considerably simpler.

Capture-HPC was found to be fairly slow, justifiable by the amount of work that
must happen to render the content in a real web browser. The issue of speed
meant that Capture-HPC has been integrated into the framework with some care,
restricting the number of URLs passed to it via classification to attempt to
limit the execution time for a set of URLs. Despite this Capture-HPC still 
remains one of the more powerful malware scanners available in the framework.


\subsection{Framework}
