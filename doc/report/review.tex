\section{Literature Review}

Previous reports in the relevant areas were researched. In this section we are
going to introduce some background knowledges and research contributions
summarised from three reports written by authoritative institutions. We will
see the importance of trending keywords in malware distribution, and the
methods and systems to measure and counter them. Approaches for malicious
website detection and analysis are also included. 

\subsection{Search-engine optimization} 

Search engine is a key tool for
exploring resources on the Internet and becomes increasingly significant in
people's daily life. Many famous search engines index web pages using PageRank
algorithm, which states that the rank of a web page depends on the quantity and
quality of incoming links, and the higher is the rank, the more likely that the
web page will apear in the top results of a search query. Although search
engine providers do not usually reveal their algorithms in calculating
PageRanks in order to reduce spammers who gaming with the system, many main
features are still well-known. For example, many webpages have their main title
stored in the URLs, therefore the URLs themself may represent a summary of the
page contents, which is a feature that can be utilised by search engines to
rise PageRank.
\paragraph{}
A report produced by University of Washington and Microsoft
Research Silicon Vally introduces a widely used search results poisoning
technique named Search Engine Optimization (SEO) as well as a system that
automatically detects them, called deSEO.\cite{deseo} They claim to be "the
first to present a systematic study of search-result poisoning attacks and how
to detect them". SEO is a process that uses varies approaches to affect a web
page such that it will apear in higher positions of a search engine's search
results, and for most of the cases, a rise of PageRank is achieved. This
process can be exploited to support illegal websites, such as malicious
websites, with some additional unethical techniques (which will be introduced
later) therefore achieves conspicuous boosts in PageRanks. This is called SEO
attacks. According to the paper, a typical SEO attack consists three servers,
which are compromised server, redirection server and exploit server. The
process of exploitation can be divided into five steps: 
\paragraph{}
\begin{enumerate} 
\item The victim submit a common request to a search engine.
\item The victim click on a search result and is navigated to a malicious web
page hosted on a compromised server.
\item The compromised server fowards the request to a redirection server.
\item The redirect server selects an exploit server then navigates the user
to it.  
\item Exploitation of the vitim's browser or scareware displayed.  
\end{enumerate}

To compromise a server, the attackers often target specific popular software
with known vunlerabilities but are still used by many servers. The example in
the report is a shopping websites management software called osCommerce, and
the attackers compromise the servers by uploading php scripts. Those scripts
are likely to have funtions that can navigate through the server's file system,
find data, and upload further files etc. When a server is compromised, SEO
pages can then be generated. Those pages contains a large number of links
connects to other SEO pages (probably on other compromised servers), as well as
contents that relate to the keywords this page's URL focuses on. Such dense
link structure and high relevant contents are the main means to elevate the
PageRank. The generated SEO pages may have clocking techniques, which means
they present different contents to search engine crawlers and victims. When the
page is visited by a search engine crawler, links to other SEO pages and
poisoned keywords are displayed, and when a victim accesses, the user's browser
is forwarded to a redirection server. 

\paragraph{}
Redirection servers are often formed
with several layers before reaching the exploit server, where they redirect
within the compromised server network. In the paper's case the domains of those
servers are very similar and according to their IPs it is determined that those
servers are hosted in physical groups. The Exploit servers in that example have
191 domains being hosted on only two IPs, and the webpage is very cautious with
incoming visitors. The page detects search engine crawlers as well as referrer,
and also blocks IPs from search engine companies.  

\paragraph{} 
The
researchers of the report crawled across the network of SEO links and finially
determined 5400 compromised domains. Each domain have an average links of
80,000 to other domains with 8,000 keywords corresponding to the same number of
SEO pages, which proves the use of dense link structure in rising PageRank.
They also provided an estimation of the number of victims affected by this
group of domains, which was produced from a dataset collected for two and a
half months. The result was that over 81,000 users were directed to exploit
pages.
\paragraph{} 
The detection of SEO attacks is not a trivial task, as
those pages usually trigger malicious actions only after strict requirements
are met, such as depending on the execution result of a JavaScript with the
input of a user's mouse action. Fortunately the report proved that study only
the URLs' structures is sufficient to complete the detections. 


\subsection{Trending-term exploitation}


\subsection{Automatic malware collection}


