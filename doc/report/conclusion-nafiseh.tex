\subsection{URL Classification}

%need to make sure it is clear that we're talking about the URLs rather than the content?
An unknown malicious URL cannot be identified by systems that use signature-based malware detection methods. As a solution, the designed system includes a classifier in order to detect unknown malicious URLs. The classifier relies on pattern matching algorithm used in the Levenshtien distance and has two classes of data; malicious URLs and clean URLs. An unknown URL enterings to the detection system will be classified based on its similarity to the URLs in the dataset. The result of the classification algorithm provides a confidence rate regarding the URL’s maliciousness. Depending on the confidence rate provided, the classifier will then classify whether the given URL should be subject to a heavyweight or lightweight scanning process. However, where the URL is already held within the database, an immediate result is returned confirming with a one hundred percent confidence rate whether or not the URL is malicious. Because a given URL does not need to be checked by multiple scanners simultaneously and unnecessarily, the classifier creates a more efficient and robust detection system. 
 
\subsection{HTML Scanning}

The HTML malware scanner specifically processes trendy keywords from a search engine and within an infected web page to detect a malicious URL. The HTML scanner has been designed based on the repetitive frequency of the trendy keywords within the main URL web page and within the contents of associated hyperlinked web pages. The scanner is efficient and can easily be extended with sophisticated techniques to produce more accurate results. 
