\Section{Literature review}
Malware is now an international word, which gained prominence following the creation and development of the internet. There are many definitions of malware. Malware is often defined as a short form of malicious software that contains malicious programs such as computer viruses and worms. With the increasing sophistication of malware attacks and their potential for causing major damage and disruption, there is an increasing need for the development of sophisticated malware detectors.\cite{Malware-tech}

Malware detectors employ a variety of malware detection techniques depending on the type of malware being addressed. In general, malware detection techniques can be classified into two groups: the first group is heuristic-based (or anomaly-based) detection that defines normal behaviour in order to determine the maliciousness of the software. The second group is signature-based detection that compares the content to the code signature of known malware in order to decide the maliciousness of a program under inspection. However, current detection techniques that are based on known malware are unable to detect a malcode unless it has been executed. Therefore, not all malwares can be detected by these two techniques alone. Hence, it is crucial to consider how we can extend the available detection tools by designing and developing a detection system capable of detecting other unknown malwares.\cite{Malware-tech}
 

In recent years, machine learning methods were introduced to the malware detection systems to extend the awareness of heuristic-based methods. The classification algorithm is one of the standard methods in machine learning that could solve the problem of detecting unknown malwares. \cite{Machin-lear}
The idea of classification algorithm is based on pattern matching. The classifier contains a set of learnt patterns and decides the maliciousness of an unknown program by comparing to these patterns. The learnt pattern set is as a result of applying learning methods on a training set that contain instances of both malicious and benign files. A recent study on integrating machine learning methods with heuristic-base methods produced positive results in detecting new malware.\cite{Machin-lear}
The file representation and feature selection methods are other machine learning techniques that greatly improve the result of malware detection systems. The latter method is concerned with patterns used for demonstrating executable files while the former tries to decrease the dimensionality of patterns that appear in the file.\cite{Machin-lear}

Polymorphic viruses are a complex type of malware that many widely used malware detection software will fail to recognise because malware detection software tends to employ Heuristic-based and signature-based algorithms. There are drawbacks to using signature-based methods because they are expensive and also slow.  The recently designed system called Intelligent Malware Detection System (IMDS) is capable of recognising both polymorphic and previously unseen malicious executable (new malware). IMDS uses data mining based classification method. The evaluation and experimental results of the IMDS illustrates its high performance against some anti-virus software such as McAfee virus scan. The architecture of the IMDS system includes three main modules; the parser, the rule generator and the malware detection part. \cite{IMDS}

A Pattern Recognition System for Malicious PDF Files Detection (academic paper)

One of the most commonly used formats for reading documents is PDF. Therefore, it is likely that an attacker uses PDF files for malicious purposes. Also, it is important to note the structure of PDF documents, which is made up of a sequence of flexible dictionary objects. This flexibility makes PDF files more attractive for an attacker because they can host a variety of applications such as java script and embedded files. Attackers mostly apply obfuscation techniques to bypass intelligent detection systems. With the aid of machine learning techniques, a new PDF detector has been designed to recognise infected PDF documents.\cite{maiorca2012pattern}
The PDF detector contains three main steps; a data retrieval module, a feature extractor module and the classifier. The system adapted K-mean clustering method to collect the set of keywords for the feature extractor module. The classifier was tested using different classification algorithms such as Naïve Bayes and SVM, with the most accurate and efficient algorithm being selected for the final design.\cite{maiorca2012pattern}
 

The signature-based method can fail to detect the obfuscation code because the obfuscation only has a minor effect on execution behaviour. Gao et al proposed the semantic based malware detection algorithm to handle obfuscation in infected documents. According to this algorithm the semantic of the program is compared to the semantic of the malware. As a result, this proposed method reduces the need for frequently updating the malware database; hence making it remarkably robust.\cite{Amom} 


Malware is continuing to grow and become more complex. An effective malware detector must be capable of detecting unknown malware as well as hidden malcode. However, malware writers analyse the techniques used in commercial malware scanners and attempt to produce new malware that can bypass the techniques adopted by such virus scanners. Therefore, an increase in unknown malware is at present innevitable.\cite{vinod2009survey}


Another study performed on malware detection methods was called BINSPECT, which provided a lightweight approach to identifying malicious web pages. This approach employed supervised learning technique for its detection and considered malware distributing techniques such as; phishing, derived-by-download and injection.\cite{eshete2012binspect}

BINSPECT used an unknown webpage as an input and classified the web page as either malicious or benign. Its classification relied on the features extracted from a given web page and the web page features were defined in three categories; features of URL, source code of webpage and social reputation features. 
For the URL features, the structure of URL (for example, number of dots ‘.’) or the length of the URL string was analyzed. The results showed the most malicious URL has abnormal length. For the web page source code, an emulated browser was used to capture the obfuscated JavaScript. Some of the features considered in the web page sources code included the number of; lines, links and suspicious functions. The extracted features were then used to collect the data for the training set and learning set. The data was then classified by applying machine-learning methods such as Logistic Regression, SVM (Support Vector Machine) and Decision Tree.\cite{eshete2012binspect}
The paper concludes its proposed method is very effective in detecting a malicious webpage with an accuracy rate exceeding 97% and that by adding the new features to the system its performance will significantly improve. However BINSPECT lacked the ability to analyse obfuscated JavaScript.\cite{eshete2012binspect} 
