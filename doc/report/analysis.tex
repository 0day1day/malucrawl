\section{Specification}

To aid in the task of designing the solution, a set of requirements have been drawn up. These requirements have been derived from the division of the task presented in the goals of the project with MOSCOW priority attached to each.

\begin{enumerate}
    \setlength{\itemsep}{1pt}
    \item The system \textbf{must} be able to collect trends
    \item The system \textbf{should} be able to collect trends from more than once source
    
    \item The system \textbf{must} be able to discover target web pages using a search engine
    \item The system \textbf{should} be able to discover target web pages using multiple different search engines
    \item The system \textbf{could} be able to discover target web pages using Twitter
    
    \item The system \textbf{must} be able to determine if a page is malicious or not
    \item The system \textbf{must} be able to control multiple different malware scanning components
    \item The system \textbf{must} use High Interaction malware Scanning
    \item The system \textbf{must} use Zero Interaction malware Scanning
    \item The system \textbf{must} use Low Interaction malware Scanning
    \item The system \textbf{should} be able to dispatch dispatch only likely candidate malicious URLs to slower High Interaction Scanners
    \item The system \textbf{should} be able to fine tune High Interaction Scanner dispatch probability
    \item The system \textbf{could} use historical speed and current status of the system to influence High Interaction Scanner dispatch probability

    \item The system \textbf{must} be able to store data determined about targets
    \item The system \textbf{must} be able to relate data about targets to the trend source and time it was discovered on
    \item The system \textbf{must} be able to generate reports
    \item The system \textbf{should} be able to show the time taken for a particular 
    \item The system \textbf{could} display reports dynamically using a web based interface
    \item The system \textbf{should} be able to monitor the status of the system
    \item The system \textbf{should} be able to monitor the status of each malware scanning component
    \item The system \textbf{could} be able to display statuses with a web interface
\end{enumerate}

\subsection{Customer Interaction}
The project was put forward by Team Cymru Research NFP, ``a specialized Internet security research firm and 501(c)3 non-profit''\cite{team-cymru}. Team Cymru maintains various real-time monitoring ``pods'' to measure the status of malware on the public Internet. The project compliments their existing offerings by incorporating social media and news trends with active web scanning.

Over the course of the project we had two meetings with Team Cymru representative Chas Tomlin: a video conference and an informal meeting.  Meetings were intended to occur at the start of the project and after each of the presentations, however only one of the post-presentation meetings occurred.

The video conference was used to introduce the team to Team Cymru and discuss the project requirements, in which the brief was expanded upon. The specification was influenced by requests to:
\begin{enumerate}
    \item A requirement to emulate a user browsing the web using a virtualized web browser also referred to as High Interaction Malware Scanning.

    \item A requirement to use simple crawling and static analysis as in Low Interaction Malware Scanners, to be determined as part of the literature search.

    \item A requirement to create a scanning modules that uses signature oriented scanning of pages and the hyper-linked files

    \item A requirement to write modules that make use of existing repositories of malicious websites such as Google's Safe Browsing API\cite{google-safe} Google's Safe Browsing API and the crowd sourced Web Of Trust.

    \item Make the code machine-independant, so as to allow the scanning modules to operate on cheap, commodity, cloud machines such as Amazon Web Services Spot Instances\cite{aws-spot} or Rackspace Cloud instances\cite{rackspace}
\end{enumerate}

Also discussed was the possibility of downloading and investigating a ``Drive by Download'' malicious software distribution tool.  The legal considerations of downloading and executing an illegal piece of software are complex and as such would require the group to obtain a detailed set of permissions before attempting such a task. Therefore a decision was made to not investigate ``Drive By Download'' toolkits as part of the project.

After the first presentation an informal meeting was held to discuss progress on the project so far with Chas:

\begin{enumerate}
    \item A suggestion to use Internet Explorer 6 running under several interfaces of Wine to act as a faster High Interaction malware scanner than virtualizing an entire operating system.
    \item A suggestion to use ClamAV as the scanning engine behind the signature oriented scanning malware scanner.
    \item Lowered the requirement of a web based reporting interface from \textbf{could} to \textbf{should}, as being able to manually query the database would be acceptable.
    \item Added the requirement to be able to monitor the system, components and workers specifically mentioning Nagios, however it was chosen to use Celery's own monitoring tools instead
\end{enumerate}

One further meeting was to be held after the second presentation, but Chas was not available.

\nomenclature{API}{Application programming interface}
