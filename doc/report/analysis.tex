%\section{Analysis \& Specification}

\section{Analysis}
%how I worked out how to do it...
\subsection{Analysis Approach}
%this is my dev approach, content here + at in design result of this.

\nomenclature{SOWN}{Southampton Open Wireless Network}
\subsection{Testing Approaches}

\subsubsection{RADIUS Peered Testing}

\subsubsection{Client Based Testing}

\nomenclature{AP}{Access Point}

\subsection{Presentation of Test Data}


\subsection{Credential Exchange Security Concerns}

\subsection{Proposed Solutions to Security Concerns}
\subsubsection{Looking-Glass}

\nomenclature{AS}{Autonomous System}
\nomenclature{BGP}{Border Gateway Protocol}


\subsubsection{Encrypted Test Schedule}


\clearpage
\section{Specification}

To aid in the task of designing the solution, a set of requirements have
been drawn up. These requirements have been derived from a combination of the
division of the task presented in the goals of the project, and the analysis
presented above, with MOSCOW priority attached to each.

\begin{enumerate}
    \setlength{\itemsep}{1pt}
    \item The system \textbf{must} generate data that indicates the status of local
802.1X authentication using appropriate test tools, where the \emph{local}
deployment references the site under test.
    \item The system \textbf{must} generate data that indicates the status of remote
access at the local eduroam deployment using appropriate test tools.
    \item The system \textbf{should} generate data that indicates the status of remote
authentication by users of the local deployment.
    \item The system \textbf{should} generate data for different methods of
authentication supported by 802.1X.
    \item The system \textbf{should} generate data to report the level of functionality
available to remote users, with reference to the JRS Tiers\cite{janet-tech}.
    \item The system \textbf{must} transfer test results from the testing nodes to a
suitable platform for storage and presentation.
    \item The system \textbf{should} allow data transfer in a way which is scalable to
the scope of all the UK eduroam sites.
    \item The system \textbf{should} transfer test results using a reliable mechanism.
    \item The system \textbf{must} store test data collected from each test site.
    \item The system \textbf{should} maintain historical data records to assist in the
diagnosis of intermittent network issues.
    \item The system \textbf{should} store the test data in a format that aids
presentation and development of future tools that use the data.
    \item The system \textbf{must} allow users to view data in a human friendly format
    \item The system \textbf{should} provide a variety of different data views, to cater
for the different users and their respective needs.
    \item The system \textbf{could} provide an API to simplify integration with other tools.
    \item The system \textbf{will not} include specific capabilities for mobile devices,
or extensive eduroam user-oriented applications.
    \item The system \textbf{must} be able to be deployed on discrete hardware devices,
with both wired and wireless interfaces, to enable easy roll-out of the system.
\end{enumerate}
\nomenclature{API}{Application programming interface}
