\Section{Conclusion}
An unknown malicious URL cannot be identified by systems that use signature-based malware detection methods. As a solution, the design system has devised a classifier to detect an unknown malicious URL. The classifier relied on patterns matching algorithm used in the Levenshtien distance. Also, it has two classes of data, malicious URLs and clean URLs. Any unknown URL entering the detection system will be classified based on its similarity to the components of the URLs in the data set. The result of classification algorithm is a figure that shows the confidence rate for the URL’s maliciousness. Depending on how satisfying is the result the classifier will send the given URL through the heavyweight scanning process or light scanning process. In the case where the URL is already in the database, the result will produce immediately with hundred percentage confidence rate. Therefore, the classifier makes a detection system more efficient and robust. Hence, any given URL does not need to be checked under all scanners unnecessarily.    
The HTML malware scanner specifically considered the relation of using trendy-term in the search engine and infected web page. The HTML scanner has been designed based on the repetitive frequency of the trendy-terms in the web page and within the contents of linked pages hyperlinked to the main web page. The scanner is efficient and can easily be extended with more sophisticated techniques to produce more accurate results. 
