\section{Evaluation}

\subsection{Evaluation of Wine Explorer}
Wine Explorer is indeed a feasible approach for high interaction malware 
detection. As Wine is a compatibility layer instead of virtual machine or 
emulator, the Internet Explorers running under it can execute in full speed, 
where the computation resource consumption is minimal compared to what 
a virtual machine or emulator require. The applications run by Wine are also 
treated as ``first class citizens''\cite{wineperformance} such that 
permissions for them are not 
restricted. The creation of a Wine prefix is also impressively fast, which 
means the overhead for honeypot reset is minimal. \\
However disadvantages still exists. We explained that the Wine prefix is treated
as a sandbox, and the reason is almost all Windows applications do not operate 
with Linux environment. Imagine if attackers predicted the Wine Explorer 
environment, then they could easily develop Windows applications which inject 
Linux shell commands, which means, they will have access to the file system 
outside of Wine prefix. Also, the suspicious behaviour detection functions of 
this program is really naive which cannot cover all situations. An example 
could 
be a downloaded virus that is able to delete itself before the scan begins. 
During that time it is sufficient to perform illegal behaviours such as 
sensitive information collection. The modification of system 
registry is also not considered in this application. According to the project 
schedule we are unable to achieve everything perfectly due to the approaching 
deadline. 

\subsection{Evaluation of Wine Explorer}
Wine Explorer is indeed a feasible approach for high interaction malware 
detection. As Wine is a compatibility layer instead of virtual machine or 
emulator, the Internet Explorers running under it can execute in full speed, 
where the computation resource consumption is minimal compared to what 
a virtual machine or emulator require. The applications run by Wine are also 
treated as ``first class citizens''\cite{wineperformance} such that 
permissions for them are not 
restricted. The creation of a Wine prefix is also impressively fast, which 
means the overhead for honeypot reset is minimal. \\
However disadvantages still exists. We explained that the Wine prefix is treated
as a sandbox, and the reason is almost all Windows applications do not operate 
with Linux environment. Imagine if attackers predicted the Wine Explorer 
environment, then they could easily develop Windows applications which inject 
Linux shell commands, which means, they will have access to the file system 
outside of Wine prefix. Also, the suspicious behaviour detection functions of 
this program is really naive which cannot cover all situations. An example 
could 
be a downloaded virus that is able to delete itself before the scan begins. 
During that time it is sufficient to perform illegal behaviours such as 
sensitive information collection. The modification of system 
registry is also not considered in this application. According to the project 
schedule we are unable to achieve everything perfectly due to the approaching 
deadline. 


\subsection{Evaluation of Wine Explorer}
Wine Explorer is indeed a feasible approach for high interaction malware 
detection. As Wine is a compatibility layer instead of virtual machine or 
emulator, the Internet Explorers running under it can execute in full speed, 
where the computation resource consumption is minimal compared to what 
a virtual machine or emulator require. The applications run by Wine are also 
treated as ``first class citizens''\cite{wineperformance} such that 
permissions for them are not 
restricted. The creation of a Wine prefix is also impressively fast, which 
means the overhead for honeypot reset is minimal. \\
However disadvantages still exists. We explained that the Wine prefix is treated
as a sandbox, and the reason is almost all Windows applications do not operate 
with Linux environment. Imagine if attackers predicted the Wine Explorer 
environment, then they could easily develop Windows applications which inject 
Linux shell commands, which means, they will have access to the file system 
outside of Wine prefix. Also, the suspicious behaviour detection functions of 
this program is really naive which cannot cover all situations. An example 
could 
be a downloaded virus that is able to delete itself before the scan begins. 
During that time it is sufficient to perform illegal behaviours such as 
sensitive information collection. The modification of system 
registry is also not considered in this application. According to the project 
schedule we are unable to achieve everything perfectly due to the approaching 
deadline. 


\subsection{Evaluation of Wine Explorer}
Wine Explorer is indeed a feasible approach for high interaction malware 
detection. As Wine is a compatibility layer instead of virtual machine or 
emulator, the Internet Explorers running under it can execute in full speed, 
where the computation resource consumption is minimal compared to what 
a virtual machine or emulator require. The applications run by Wine are also 
treated as ``first class citizens''\cite{wineperformance} such that 
permissions for them are not 
restricted. The creation of a Wine prefix is also impressively fast, which 
means the overhead for honeypot reset is minimal. \\
However disadvantages still exists. We explained that the Wine prefix is treated
as a sandbox, and the reason is almost all Windows applications do not operate 
with Linux environment. Imagine if attackers predicted the Wine Explorer 
environment, then they could easily develop Windows applications which inject 
Linux shell commands, which means, they will have access to the file system 
outside of Wine prefix. Also, the suspicious behaviour detection functions of 
this program is really naive which cannot cover all situations. An example 
could 
be a downloaded virus that is able to delete itself before the scan begins. 
During that time it is sufficient to perform illegal behaviours such as 
sensitive information collection. The modification of system 
registry is also not considered in this application. According to the project 
schedule we are unable to achieve everything perfectly due to the approaching 
deadline. 


\subsection{Capture-HPC}
Although Capture-HPC is a very capable malware scanning component, there were
numerous difficulties involved setting it up. Compiling the Windows kernel
drivers for the client was a very complex process, and will need to be repeated
to generate installers for other Windows OSs. The ECS specific modifications
that required two sets of RPC over Message Queues to work also caused considerable
delay to the integration of Capture-HPC into the framework, but means that some
level of network security is maintained separating the hosts running the rest of
the framework from the host running Capture-HPC.

\subsection{Malware Lists}
Although very fast in terms of URLs scanned per second, the components of the system using Malware Lists give very limited results about those URLs. This is not a severe problem as the intention of these components is to avoid scanning high traffic, or high trustworthy sites such as Google or Wikipedia.

\paragraph{}
Difficulties involved with interacting with the counter-intuitive, and in some cases defective, Celery Batches API resulted in more time expended compared to that originally expected. Fortunately it was possible to fix the issues resulting in a defective Batches API and submit those upstream.

\subsection{Framework}
The distributed framework allowed the system to control multiple independent malware scanning sub-systems.  The distributed framework is the best solution for the project because multiple systems can be scanning concurrently on a vast array of hardware. Celery allowed the framework to define Tasks and describe the way in which data must flow through the system: stating which Tasks depended on which other Tasks without directly exposing concepts of Queues and Locks.

\paragraph{}
However while the ability to distribute the system across multiple machines offered a significant performance benefit it also reduced the rate of development and caused difficulties during integration because multi-threaded programming concepts are generally counter-intuitive and difficult to reason about for those not used to the concept.


