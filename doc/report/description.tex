\section{Introduction}

%chris's intro stuff
\subsection{Emma Watson and Malware}
On September 10th 2012, MacAfee released a report detailing a list of the top 20
celebrities that it believed were the most risky to search for on the internet\cite{mac-watson}.
The report specifies a number of threats from risky web sites, such as malware,
phishing, and spam, and determined that Emma Watson was the most risky celebrity
to search for on the web with more than 12.6\% chance of a site being malicious
when searching for her name combined with phrases such as ``free download''.

The hypothesis suggested in the MacAfee report is that trending topics have
malicious web pages crafted specifically to target the topic, and this is the
hypothesis that will be investigated in the project.

Before investigating if malware is targeted at trending topics, it is necessary to
investigate exactly what constitutes a trending topic. ``Trending topic'' is a term
that originates from Twitter, where statistics are collected on users tweeting
with a given hashtag and used to determine popular hashtags, presented in the
user interface a ``trending topic''. When discussing trending topics in this
report, other sources of trends and popular subjects will also be considered as
trending topics.

To make it possible to investigate the hypothesis proposed above a framework
will be designed and built to automatically gather trends, search for URLs
relating to the trends, and then attempt to detect whether the URL is hosting
malicious content. The data will then be collected into a database, allowing the
hypothesis to be tested. Results will be reported using a web interface that
will also be capable of monitoring the status of the malware scanning system.

%nafiseh's intro stuff
\section{Introduction}

\subsection{Problem Definition}

The continually increasing variety of known and unknown malwares make malware detection a difficult problem to solve. As malware detection systems become more intelligent and sophisticated, malware writers are also developing more advanced methods. Some anti-virus software is ineffective in terms of detecting unknown malware. This project seeks to design a malware detection system capable of identifying web pages containing malicious content for a given contextual topic as represented by trending search keywords. 
Also, this research involved emulating a user browsing for these keywords. The challenging part of this project arose in seeking to define a malicious web page, as there are many variations of an infected webpage depending on the type of malware. 
Some types of malware directs the user to an illegal website by clicking on the URL. A large group of malware are designed with a commercial objective in mind. A webpage that has poor content, offers no goods or services and is used for the purpose of directing users to unsolicited advertisements is an example of commercial malware. Whilst other web pages may look perfectly fine at first glance but in fact contain malicious functions, for example by using JavaScript functions. In the case of the latter, the detection is even more difficult because the malware is a type of obfuscation malware that is hidden from the user. Moreover, malwares could come from a variety of different sources.
The examples above highlight the difficulties of this project. Furthermore, whilst a great deal of research has been performed on the subject of malware detection and the effectiveness of various detection methods, such research specifically considered a specifically defined type of malware and analysed the methods used for that particular malware. 
In seeking to propose a malware detection system capable of identifying web pages containing malicious content based on trending search keywords, this paper gives consideration to the fact that different variations of malware exist. 

